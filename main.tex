\documentclass[letterpaper, 10 pt, conference]{ieeeconf}
\IEEEoverridecommandlockouts                              
\overrideIEEEmargins
% See the \addtolength command later in the file to balance the column lengths
% on the last page of the document



% The following packages can be found on http:\\www.ctan.org
%\usepackage{graphics} % for pdf, bitmapped graphics files
%\usepackage{epsfig} % for postscript graphics files
%\usepackage{mathptmx} % assumes new font selection scheme installed
%\usepackage{times} % assumes new font selection scheme installed
%\usepackage{amsmath} % assumes amsmath package installed
%\usepackage{amssymb}  % assumes amsmath package installed

\title{\LARGE \bf
Aerial Robotics Kharagpur Task 1.1
}


\author{Lakshay Gupta  
}


\begin{document}



\maketitle 
\thispagestyle{empty}
\pagestyle{empty}


%%%%%%%%%%%%%%%%%%%%%%%%%%%%%%%%%%%%%%%%%%%%%%%%%%%%%%%%%%%%%%%%%%%%%%%%%%%%%%%%
\begin{abstract}
\textbf{The task is to detect edges in the image table.png so that Luna is prevented from falling off the table's edge}


I have used a sobel edge operator from scratch to detect edges in the image. The Sobel edge detector uses a pair of 3 x 3 convolution masks, one estimating gradient in the x-direction and the other estimating gradient in y–direction. The Sobel detector is incredibly sensitive to noise in pic- tures, it effectively highlight them as edges. For the detection of edge lines I have used the hough transform. It converts shapes such as lines and circles into mathematical representations in a parameter space, making it easier to identify them even if they're broken or obscured. Some applications include object tracking and recognition, biometrics including hand detection models etc.
\end{abstract}




%%%%%%%%%%%%%%%%%%%%%%%%%%%%%%%%%%%%%%%%%%%%%%%%%%%%%%%%%%%%%%%%%%%%%%%%%%%%%%%%
\section{INTRODUCTION}

The problem statement requires to make an algorithm that can detect edges in the image table.png and which prevents Luna from falling off the table.To perform this task we can use many edge detection tools such as the sobel, canny or laplacian based edge detectors. I have used sobel as it is relatively simpler to operate. To construct edge lines from edge.png, I have used the hough transform. I have tried various threshold levels and also while binarizing the image i have set a threshold level but still my code detects an extra edge line which is a false edge.

\section{Problem Statement}

The problem statement required 2 things to be done. The first was to detect edges in the image table.png and then detect edge lines from the subsequent edge.png. For the former task I have used the sobel operator which uses a pair of 3 x 3 kernels, one for the x-direction and one for the y directions, the kernels are as follows:
    \centering
    \includegraphics[width=0.75\linewidth]{image1.png}
    \label{}

When using Sobel Edge Detection, the image is processed in the X and Y directions separately first, and then combined together to form a new image which represents the sum of the X and Y edges of the image.
    \centering
    \includegraphics[width=1\linewidth]{image.png}
Next we use the Hough transform to convert well differentiated edge lines from all the detected edges. It uses a voting mechanism to identify bad examples of objects inside a given class of forms. This voting mechanism is carried out in parameter space. Object candidates are produced as local maxima in an accumulator space using the HT algorithm.
    \centering
    \includegraphics[width=1\linewidth]{image.png}


\section{INITIAL ATTEMPTS}

Initial attempts include using of the opencv library to apply the sobel operator and hough transform. 


\section{Final Approach}
\textbf
The Final code for sobel edge detection:
import numpy as np
import cv2
import matplotlib.pyplot as plt

def convol(image, kernel):
    # dimensions of the image and kernel
    image_h, image_w = image.shape
    kernel_size = kernel.shape[0]
    print (kernel.shape[0])
    kernel_radius = kernel_size // 2
    
    # Initializing the result matrix
    result = np.zeros_like(image)
    
    # Padding the image to handle borders
    padded_image = np.pad(image, ((kernel_radius, kernel_radius), (kernel_radius, kernel_radius)), mode='constant')
    
    # convolution
    for y in range(kernel_radius, image_h + kernel_radius):
        for x in range(kernel_radius, image_w + kernel_radius):
            # Extracting the region of interest (ROI)
            roi = padded_image[y - kernel_radius:y + kernel_radius + 1, x - kernel_radius:x + kernel_radius + 1]
            # Computing the convolution sum
            result[y - kernel_radius, x - kernel_radius] = np.sum(roi * kernel)
    
    return result

def sobel(image, threshold=100):
    # Converting image to grayscale
    gray_scale = cv2.cvtColor(image, cv2.COLOR_BGR2GRAY)
    gray_scale = cv2.GaussianBlur(gray_scale,(3,3),0)
    
    # Defining Sobel kernels for horizontal and vertical edge detection
    sobel_x = np.array([[-1, 0, 1],
                        [-2, 0, 2],
                        [-1, 0, 1]])
    
    sobel_y = np.array([[-1, -2, -1],
                        [0, 0, 0],
                        [1, 2, 1]])
    
    # Convolving the image with Sobel kernels
    grad_x = convol(gray_scale, sobel_x)
    grad_y = convol(gray_scale, sobel_y)
    
    # Computing the gradient magnitude
    grad_mag = np.sqrt(np.square(grad_x) + np.square(grad_y))
    
    # Normalizing gradient magnitude to [0, 255]
    grad_mag *= 255.0 / grad_mag.max()
    grad_mag[grad_mag < threshold] = 0

    return grad_mag
    

# Loading image
image = cv2.imread('Photos/table.png')

# Resizing the image to improve processing speed
image = cv2.resize(image, (500, 500))
ima_blur = cv2.GaussianBlur(image,(3,3),0)

# Performing edge detection
edges = sobel(ima_blur, threshold=150)

# Saving the detected edges as a PNG image
cv2.imwrite('edge.png', edges)


# Displaying the original and edge-detected images using Matplotlib
plt.figure(figsize=(10, 5))

plt.subplot(1, 2, 1)
plt.imshow(cv2.cvtColor(image, cv2.COLOR_BGR2RGB))
plt.title('Original Image')
plt.axis('off')

plt.subplot(1, 2, 2)
plt.imshow(edges, cmap='gray')
plt.title('Edges Detected')
plt.axis('off')

plt.show()




Final code for Hough transform:

import numpy as np
import cv2

def binarise(image, threshold=128):
    # Converting image to grayscale
    gray_image = cv2.cvtColor(image, cv2.COLOR_BGR2GRAY)
    
    # Binarising image using the given threshold
    _, binary_image = cv2.threshold(gray_image, threshold, 255, cv2.THRESH_BINARY)
    
    return binary_image

def hough(image, theta_res=1, rho_res=1, threshold=100):
    # Defining the range of theta (0 to 180 degrees) and rho (-max_distance to max_distance)
    theta_range = np.deg2rad(np.arange(0, 180, theta_res))
    max_distance = int(np.sqrt(image.shape[0]**2 + image.shape[1]**2))
    rho_range = np.arange(-max_distance, max_distance + 1, rho_res)
    
    # Initializing the Hough accumulator
    accumulator = np.zeros((len(rho_range), len(theta_range)), dtype=np.uint64)
    
    # Finding edge pixels in the image
    edge_pixels = np.argwhere(image > 0)
    
    # Loop over each edge pixel
    for y, x in edge_pixels:
        # Loop over each theta value
        for theta_idx, theta in enumerate(theta_range):
            rho = int(x * np.cos(theta) + y * np.sin(theta))
            rho_idx = np.argmin(np.abs(rho_range - rho))
            accumulator[rho_idx, theta_idx] += 1
    
    # Finding indices of accumulator cells above the threshold
    rho_idxs, theta_idxs = np.where(accumulator >= threshold)
    
    # Extracting rho and theta values for detected lines
    rhos = rho_range[rho_idxs]
    thetas = np.rad2deg(theta_range[theta_idxs])
    
    return rhos, thetas

def lines(image, rhos, thetas):
    for rho, theta in zip(rhos, thetas):
        a = np.cos(np.deg2rad(theta))
        b = np.sin(np.deg2rad(theta))
        x0 = a * rho
        y0 = b * rho
        x1 = int(x0 + 1000 * (-b))
        y1 = int(y0 + 1000 * (a))
        x2 = int(x0 - 1000 * (-b))
        y2 = int(y0 - 1000 * (a))
        cv2.line(image, (x1, y1), (x2, y2), (0, 0, 255), 2)

# Reading the edge image
edge_image = cv2.imread('edge.png')


# Binarising the edge image
binary_edge_image = binarise(edge_image, threshold=231)  # Adjustable threshold

# Applying Hough Transform
rhos, thetas = hough(binary_edge_image, theta_res=1, rho_res=1, threshold=101)  # Adjustable threshold

# Reading original image
original_image = cv2.imread('Photos/table.png')
original_image = cv2.resize(original_image, (500, 500))


# Drawing detected lines on the original image
lines_image = original_image.copy()
lines(lines_image, rhos, thetas)

# Saving the final result
cv2.imwrite('detected_lines.png', lines_image)

# Displaying the final result
cv2.imshow('Detected Lines', lines_image)
cv2.waitKey(0)
cv2.destroyAllWindows()


\section{Results and Observation}


    \centering
    \includegraphics[width=1\linewidth]{image2.png}
As evident from the above images, the sobel operator has detected various edges where it can find intensity difference going from the x or y direction.
    \centering
    \includegraphics[width=0.5\linewidth]{image3.png}
    
The hough transform has detected edge lines as indicated by red lines.



\section{Future Work}

One major problem with the the sobel operator is it is very sensitive to noise and intesity changes, as a result it can detect many false edges which can hamper the ouput.
To tackle this problem gaussian blur is used to minimise noise in the image and thresholding is used to filter out intensity changes. Still the algorithm does not seem very efficient as the resultant output gives out 2 edge lines as opposed to only 1 that is necessary. Thresholding is used at several places including when binarising the image.

\section*{CONCLUSION}
The sobel operator and the hough transform are some of the most widely used algorithms for edge detection algorithms. For this task it is particularly useful as the robot Luna can detect the table edge and either stop its movement or deflect its direction to avoid falling.

\begin{thebibliography}{99} 
\bibitem{c1}“Sobel Operator,” www.tutorialspoint.com. [Online]. Available: https://www.tutorialspoint.com/dip/sobel_operator.htm. 

\bibitem{c2} https://www.projectrhea.org/rhea/index.php/An_Implementation_of_Sobel_Edge_Detection/

\bibitem{c3} https://www.analyticsvidhya.com/blog/2022/06/a-complete-guide-on-hough-transform/#:~:text=Hough%20Transform%20is%20a%20computer,they're%20broken%20or%20obscured.

\end{thebibliography}

\end{document}
